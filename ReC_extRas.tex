% Options for packages loaded elsewhere
\PassOptionsToPackage{unicode}{hyperref}
\PassOptionsToPackage{hyphens}{url}
%
\documentclass[
]{book}
\title{ReCentering Psych Stats: extRas}
\author{Lynette H. Bikos, PhD, ABPP}
\date{2021-12-03}

\usepackage{amsmath,amssymb}
\usepackage{lmodern}
\usepackage{iftex}
\ifPDFTeX
  \usepackage[T1]{fontenc}
  \usepackage[utf8]{inputenc}
  \usepackage{textcomp} % provide euro and other symbols
\else % if luatex or xetex
  \usepackage{unicode-math}
  \defaultfontfeatures{Scale=MatchLowercase}
  \defaultfontfeatures[\rmfamily]{Ligatures=TeX,Scale=1}
\fi
% Use upquote if available, for straight quotes in verbatim environments
\IfFileExists{upquote.sty}{\usepackage{upquote}}{}
\IfFileExists{microtype.sty}{% use microtype if available
  \usepackage[]{microtype}
  \UseMicrotypeSet[protrusion]{basicmath} % disable protrusion for tt fonts
}{}
\makeatletter
\@ifundefined{KOMAClassName}{% if non-KOMA class
  \IfFileExists{parskip.sty}{%
    \usepackage{parskip}
  }{% else
    \setlength{\parindent}{0pt}
    \setlength{\parskip}{6pt plus 2pt minus 1pt}}
}{% if KOMA class
  \KOMAoptions{parskip=half}}
\makeatother
\usepackage{xcolor}
\IfFileExists{xurl.sty}{\usepackage{xurl}}{} % add URL line breaks if available
\IfFileExists{bookmark.sty}{\usepackage{bookmark}}{\usepackage{hyperref}}
\hypersetup{
  pdftitle={ReCentering Psych Stats: extRas},
  pdfauthor={Lynette H. Bikos, PhD, ABPP},
  hidelinks,
  pdfcreator={LaTeX via pandoc}}
\urlstyle{same} % disable monospaced font for URLs
\usepackage{color}
\usepackage{fancyvrb}
\newcommand{\VerbBar}{|}
\newcommand{\VERB}{\Verb[commandchars=\\\{\}]}
\DefineVerbatimEnvironment{Highlighting}{Verbatim}{commandchars=\\\{\}}
% Add ',fontsize=\small' for more characters per line
\usepackage{framed}
\definecolor{shadecolor}{RGB}{248,248,248}
\newenvironment{Shaded}{\begin{snugshade}}{\end{snugshade}}
\newcommand{\AlertTok}[1]{\textcolor[rgb]{0.94,0.16,0.16}{#1}}
\newcommand{\AnnotationTok}[1]{\textcolor[rgb]{0.56,0.35,0.01}{\textbf{\textit{#1}}}}
\newcommand{\AttributeTok}[1]{\textcolor[rgb]{0.77,0.63,0.00}{#1}}
\newcommand{\BaseNTok}[1]{\textcolor[rgb]{0.00,0.00,0.81}{#1}}
\newcommand{\BuiltInTok}[1]{#1}
\newcommand{\CharTok}[1]{\textcolor[rgb]{0.31,0.60,0.02}{#1}}
\newcommand{\CommentTok}[1]{\textcolor[rgb]{0.56,0.35,0.01}{\textit{#1}}}
\newcommand{\CommentVarTok}[1]{\textcolor[rgb]{0.56,0.35,0.01}{\textbf{\textit{#1}}}}
\newcommand{\ConstantTok}[1]{\textcolor[rgb]{0.00,0.00,0.00}{#1}}
\newcommand{\ControlFlowTok}[1]{\textcolor[rgb]{0.13,0.29,0.53}{\textbf{#1}}}
\newcommand{\DataTypeTok}[1]{\textcolor[rgb]{0.13,0.29,0.53}{#1}}
\newcommand{\DecValTok}[1]{\textcolor[rgb]{0.00,0.00,0.81}{#1}}
\newcommand{\DocumentationTok}[1]{\textcolor[rgb]{0.56,0.35,0.01}{\textbf{\textit{#1}}}}
\newcommand{\ErrorTok}[1]{\textcolor[rgb]{0.64,0.00,0.00}{\textbf{#1}}}
\newcommand{\ExtensionTok}[1]{#1}
\newcommand{\FloatTok}[1]{\textcolor[rgb]{0.00,0.00,0.81}{#1}}
\newcommand{\FunctionTok}[1]{\textcolor[rgb]{0.00,0.00,0.00}{#1}}
\newcommand{\ImportTok}[1]{#1}
\newcommand{\InformationTok}[1]{\textcolor[rgb]{0.56,0.35,0.01}{\textbf{\textit{#1}}}}
\newcommand{\KeywordTok}[1]{\textcolor[rgb]{0.13,0.29,0.53}{\textbf{#1}}}
\newcommand{\NormalTok}[1]{#1}
\newcommand{\OperatorTok}[1]{\textcolor[rgb]{0.81,0.36,0.00}{\textbf{#1}}}
\newcommand{\OtherTok}[1]{\textcolor[rgb]{0.56,0.35,0.01}{#1}}
\newcommand{\PreprocessorTok}[1]{\textcolor[rgb]{0.56,0.35,0.01}{\textit{#1}}}
\newcommand{\RegionMarkerTok}[1]{#1}
\newcommand{\SpecialCharTok}[1]{\textcolor[rgb]{0.00,0.00,0.00}{#1}}
\newcommand{\SpecialStringTok}[1]{\textcolor[rgb]{0.31,0.60,0.02}{#1}}
\newcommand{\StringTok}[1]{\textcolor[rgb]{0.31,0.60,0.02}{#1}}
\newcommand{\VariableTok}[1]{\textcolor[rgb]{0.00,0.00,0.00}{#1}}
\newcommand{\VerbatimStringTok}[1]{\textcolor[rgb]{0.31,0.60,0.02}{#1}}
\newcommand{\WarningTok}[1]{\textcolor[rgb]{0.56,0.35,0.01}{\textbf{\textit{#1}}}}
\usepackage{longtable,booktabs,array}
\usepackage{calc} % for calculating minipage widths
% Correct order of tables after \paragraph or \subparagraph
\usepackage{etoolbox}
\makeatletter
\patchcmd\longtable{\par}{\if@noskipsec\mbox{}\fi\par}{}{}
\makeatother
% Allow footnotes in longtable head/foot
\IfFileExists{footnotehyper.sty}{\usepackage{footnotehyper}}{\usepackage{footnote}}
\makesavenoteenv{longtable}
\usepackage{graphicx}
\makeatletter
\def\maxwidth{\ifdim\Gin@nat@width>\linewidth\linewidth\else\Gin@nat@width\fi}
\def\maxheight{\ifdim\Gin@nat@height>\textheight\textheight\else\Gin@nat@height\fi}
\makeatother
% Scale images if necessary, so that they will not overflow the page
% margins by default, and it is still possible to overwrite the defaults
% using explicit options in \includegraphics[width, height, ...]{}
\setkeys{Gin}{width=\maxwidth,height=\maxheight,keepaspectratio}
% Set default figure placement to htbp
\makeatletter
\def\fps@figure{htbp}
\makeatother
\setlength{\emergencystretch}{3em} % prevent overfull lines
\providecommand{\tightlist}{%
  \setlength{\itemsep}{0pt}\setlength{\parskip}{0pt}}
\setcounter{secnumdepth}{5}
\usepackage{booktabs}
\ifLuaTeX
  \usepackage{selnolig}  % disable illegal ligatures
\fi
\usepackage[]{natbib}
\bibliographystyle{plainnat}

\usepackage{amsthm}
\newtheorem{theorem}{Theorem}[chapter]
\newtheorem{lemma}{Lemma}[chapter]
\newtheorem{corollary}{Corollary}[chapter]
\newtheorem{proposition}{Proposition}[chapter]
\newtheorem{conjecture}{Conjecture}[chapter]
\theoremstyle{definition}
\newtheorem{definition}{Definition}[chapter]
\theoremstyle{definition}
\newtheorem{example}{Example}[chapter]
\theoremstyle{definition}
\newtheorem{exercise}{Exercise}[chapter]
\theoremstyle{definition}
\newtheorem{hypothesis}{Hypothesis}[chapter]
\theoremstyle{remark}
\newtheorem*{remark}{Remark}
\newtheorem*{solution}{Solution}
\begin{document}
\maketitle

{
\setcounter{tocdepth}{1}
\tableofcontents
}
\hypertarget{book-cover}{%
\chapter*{BOOK COVER}\label{book-cover}}
\addcontentsline{toc}{chapter}{BOOK COVER}

\includegraphics{images/ReCenterPsychStats-extRas-bookcover.png}
\# Preface \{-\}

\textbf{If you are viewing this document, you should know that this is a book-in-progress. Early drafts are released for the purpose teaching my classes and gaining formative feedback from a host of stakeholders. The document was last updated on 03 Dec 2021}. Emerging volumes on other statistics are posted on the \href{https://lhbikos.github.io/BikosRVT/ReCenter.html}{ReCentering Psych Stats} page at my research team's website.

\href{https://spu.hosted.panopto.com/Panopto/Pages/Viewer.aspx?id=c932455e-ef06-444a-bdca-acf7012d759a}{Screencasted Lecture Link}

To \emph{center} a variable in regression means to set its value at zero and interpret all other values in relation to this reference point. Regarding race and gender, researchers often center male and White at zero. Further, it is typical that research vignettes in statistics textbooks are similarly seated in a White, Western (frequently U.S.), heteronormative, framework. The purpose of this project is to create a set of open educational resources (OER) appropriate for doctoral and post-doctoral training that contribute to a socially responsive pedagogy -- that is, it contributes to justice, equity, diversity, and inclusion.

Statistics training in doctoral programs are frequently taught with fee-for-use programs (e.g., SPSS/AMOS, SAS, MPlus) that may not be readily available to the post-doctoral professional. In recent years, there has been an increase and improvement in R packages (e.g., \emph{psych}, \emph{lavaan}) used for in analyses common to psychological research. Correspondingly, many graduate programs are transitioning to statistics training in R (free and open source). This is a challenge for post-doctoral psychologists who were trained with other software. This OER will offer statistics training with R and be freely available (specifically in a GitHub respository and posted through GitHub Pages) under a Creative Commons Attribution - Non Commercial - Share Alike license {[}CC BY-NC-SA 4.0{]}.

Training models for doctoral programs in HSP are commonly scholar-practitioner, scientist-practitioner, or clinical-scientist. An emerging model, the \emph{scientist-practitioner-advocacy} training model incorporates social justice advocacy so that graduates are equipped to recognize and address the sociocultural context of oppression and unjust distribution of resources and opportunities \citep{mallinckrodt_scientist-practitioner-advocate_2014}. In statistics textbooks, the use of research vignettes engages the learner around a tangible scenario for identifying independent variables, dependent variables, covariates, and potential mechanisms of change. Many students recall examples in Field's \citeyearpar{field_discovering_2012} popular statistics text: Viagra to teach one-way ANOVA, beer goggles for two-way ANOVA, and bushtucker for repeated measures. What if the research vignettes were more socially responsive?

In this OER, research vignettes will be from recently published articles where:

\begin{itemize}
\tightlist
\item
  the author's identity is from a group where scholarship is historically marginalized (e.g., BIPOC, LGBTQ+, LMIC{[}low-middle income countries{]}),
\item
  the research is responsive to issues of justice, equity, inclusion, diversity,
\item
  the lesson's statistic is used in the article, and
\item
  there is sufficient information in the article to simulate the data for the chapter example(s) and practice problem(s); or it is publicly available.
\end{itemize}

In training for multicultural competence, the saying, ``A fish doesn't know that it's wet'' is often used to convey the notion that we are often unaware of our own cultural characteristics. In recent months and years, there has been an increased awakening to the institutional and systemic racism that our systems are perpetuating. Queuing from the water metaphor, I am hopeful that a text that is recentered in the ways I have described can contribute to \emph{changing the water} in higher education and in the profession of psychology.

\hypertarget{copyright-with-open-access}{%
\section*{Copyright with Open Access}\label{copyright-with-open-access}}
\addcontentsline{toc}{section}{Copyright with Open Access}

This book is published under a a Creative Commons Attribution-NonCommercial-ShareAlike 4.0 International License. This means that this book can be reused, remixed, retained, revised and redistributed (including commercially) as long as appropriate credit is given to the authors. If you remix, or modify the original version of this open textbook, you must redistribute all versions of this open textbook under the same license - CC BY-SA.

A \href{https://github.com/lhbikos/ReC_Psychometrics}{GitHub open-source repository} contains all of the text and source code for the book, including data and images.

\hypertarget{acknowledgements}{%
\chapter*{ACKNOWLEDGEMENTS}\label{acknowledgements}}
\addcontentsline{toc}{chapter}{ACKNOWLEDGEMENTS}

As a doctoral student at the University of Kansas (1992-2005), I learned that ``a foreign language'' was required for graduation. \emph{Please note that as one who studies the intersections of global, vocational, and sustainable psychology, I regret that I do not have language skills beyond English.} This could have been met with credit from high school my rural, mid-Missouri high school did not offer such classes. This requirement would have typically been met with courses taken during an undergraduate program -- but my non-teaching degree in the University of Missouri's School of Education was exempt from this. The requirement could have also been met with a computer language (fortran, C++) -- I did not have any of those either. There was a tiny footnote on my doctoral degree plan that indicated that a 2-credit course, ``SPSS for Windows'' would substitute for the language requirement. Given that it was taught by my one of my favorite professors, I readily signed up. As it turns out, Samuel B. Green, PhD, was using the course to draft chapters in the textbook \citep{green_using_2014} that has been so helpful for so many. Unfortunately, Drs. Green (1947 - 2018) and Salkind (2947 - 2017) are no longer with us. I have worn out numerous versions of their text. Another favorite text of mine was Dr.~Barbara Byrne's \citeyearpar{byrne_structural_2016}, ``Structural Equation Modeling with AMOS.'' I loved the way she worked through each problem and paired it with a published journal article, so that the user could see how the statistical evaluation fit within the larger project/article. I took my tea-stained text with me to a workshop she taught at APA and was proud of the signature she added to it (a little catfur might have fallen out). Dr.~Byrne created SEM texts for a number of statistical programs (e.g., LISREL, EQS, MPlus). As I was learning R, I wrote Dr.~Byrne, asking if she had an edition teaching SEM/CFA with R. She promptly wrote back, saying that she did not have the bandwidth to learn a new statistics package. We lost Dr.~Byrne in December 2020. I am so grateful to these role models for their contributions to my statistical training. I am also grateful for the doctoral students who have taken my courses and are continuing to provide input for how to improve the materials.

The inspiration for training materials that re*center statistics and research methods came from the \href{https://www.academics4blacklives.com/}{Academics for Black Survival and Wellness Initiative}. This project, co-founded by Della V. Mosley, Ph.D., and Pearis L. Bellamy, M.S., made clear the necessity and urgency for change in higher education and the profession of psychology.

At very practical levels, I am indebted to SPU's Library, and more specifically, SPU's Education, Technology, and Media Department. Assistant Dean for Instructional Design and Emerging Technologies, R. John Robertson, MSc, MCS, has offered unlimited consultation, support, and connection. Senior Instructional Designer in Graphics \& Illustrations, Dominic Wilkinson, designed the logo and bookcover. Psychology and Scholarly Communications Librarian, Kristin Hoffman, MLIS, has provided consultation on topics ranging from OERS to citations. I am alo indebted to Associate Vice President, Teaching and Learning at Kwantlen Polytechnic University, Rajiv Jhangiani, PhD. Dr.~Jhangiani's text \citeyearpar{jhangiani_research_2019} was the first OER I ever used and I was grateful for his encouraging conversation.

Financial support for this project has been provided the following grants:

\begin{itemize}
\tightlist
\item
  \emph{Call to Action on Equity, Inclusion, Diversity, Justice, and Social Responsivity Request for Proposals} grant from the Association of Psychology Postdoctoral and Internship Centers (2021-2022).
\item
  \emph{Diversity Seed Grant}, Office of Inclusive Excellence and Advisory Council for Diversity and Reconciliation (ACDR), Seattle Pacific University.
\item
  \emph{ETM Open Textbook \& OER Development Funding}, Office of Education, Technology, \& Media, Seattle Pacific University.
\end{itemize}

\hypertarget{hello-bookdown}{%
\chapter{Hello bookdown}\label{hello-bookdown}}

All chapters start with a first-level heading followed by your chapter title, like the line above. There should be only one first-level heading (\texttt{\#}) per .Rmd file.

\hypertarget{a-section}{%
\section{A section}\label{a-section}}

All chapter sections start with a second-level (\texttt{\#\#}) or higher heading followed by your section title, like the sections above and below here. You can have as many as you want within a chapter.

\hypertarget{an-unnumbered-section}{%
\subsection*{An unnumbered section}\label{an-unnumbered-section}}
\addcontentsline{toc}{subsection}{An unnumbered section}

Chapters and sections are numbered by default. To un-number a heading, add a \texttt{\{.unnumbered\}} or the shorter \texttt{\{-\}} at the end of the heading, like in this section.

\hypertarget{cross}{%
\chapter{Cross-references}\label{cross}}

Cross-references make it easier for your readers to find and link to elements in your book.

\hypertarget{chapters-and-sub-chapters}{%
\section{Chapters and sub-chapters}\label{chapters-and-sub-chapters}}

There are two steps to cross-reference any heading:

\begin{enumerate}
\def\labelenumi{\arabic{enumi}.}
\tightlist
\item
  Label the heading: \texttt{\#\ Hello\ world\ \{\#nice-label\}}.

  \begin{itemize}
  \tightlist
  \item
    Leave the label off if you like the automated heading generated based on your heading title: for example, \texttt{\#\ Hello\ world} = \texttt{\#\ Hello\ world\ \{\#hello-world\}}.
  \item
    To label an un-numbered heading, use: \texttt{\#\ Hello\ world\ \{-\#nice-label\}} or \texttt{\{\#\ Hello\ world\ .unnumbered\}}.
  \end{itemize}
\item
  Next, reference the labeled heading anywhere in the text using \texttt{\textbackslash{}@ref(nice-label)}; for example, please see Chapter \ref{cross}.

  \begin{itemize}
  \tightlist
  \item
    If you prefer text as the link instead of a numbered reference use: \protect\hyperlink{cross}{any text you want can go here}.
  \end{itemize}
\end{enumerate}

\hypertarget{captioned-figures-and-tables}{%
\section{Captioned figures and tables}\label{captioned-figures-and-tables}}

Figures and tables \emph{with captions} can also be cross-referenced from elsewhere in your book using \texttt{\textbackslash{}@ref(fig:chunk-label)} and \texttt{\textbackslash{}@ref(tab:chunk-label)}, respectively.

See Figure \ref{fig:nice-fig}.

\begin{Shaded}
\begin{Highlighting}[]
\FunctionTok{par}\NormalTok{(}\AttributeTok{mar =} \FunctionTok{c}\NormalTok{(}\DecValTok{4}\NormalTok{, }\DecValTok{4}\NormalTok{, .}\DecValTok{1}\NormalTok{, .}\DecValTok{1}\NormalTok{))}
\FunctionTok{plot}\NormalTok{(pressure, }\AttributeTok{type =} \StringTok{\textquotesingle{}b\textquotesingle{}}\NormalTok{, }\AttributeTok{pch =} \DecValTok{19}\NormalTok{)}
\end{Highlighting}
\end{Shaded}

\begin{figure}

{\centering \includegraphics[width=0.8\linewidth]{ReC_extRas_files/figure-latex/nice-fig-1} 

}

\caption{Here is a nice figure!}\label{fig:nice-fig}
\end{figure}

Don't miss Table \ref{tab:nice-tab}.

\begin{Shaded}
\begin{Highlighting}[]
\NormalTok{knitr}\SpecialCharTok{::}\FunctionTok{kable}\NormalTok{(}
  \FunctionTok{head}\NormalTok{(pressure, }\DecValTok{10}\NormalTok{), }\AttributeTok{caption =} \StringTok{\textquotesingle{}Here is a nice table!\textquotesingle{}}\NormalTok{,}
  \AttributeTok{booktabs =} \ConstantTok{TRUE}
\NormalTok{)}
\end{Highlighting}
\end{Shaded}

\begin{table}

\caption{\label{tab:nice-tab}Here is a nice table!}
\centering
\begin{tabular}[t]{rr}
\toprule
temperature & pressure\\
\midrule
0 & 0.0002\\
20 & 0.0012\\
40 & 0.0060\\
60 & 0.0300\\
80 & 0.0900\\
\addlinespace
100 & 0.2700\\
120 & 0.7500\\
140 & 1.8500\\
160 & 4.2000\\
180 & 8.8000\\
\bottomrule
\end{tabular}
\end{table}

\hypertarget{parts}{%
\chapter{Parts}\label{parts}}

You can add parts to organize one or more book chapters together. Parts can be inserted at the top of an .Rmd file, before the first-level chapter heading in that same file.

Add a numbered part: \texttt{\#\ (PART)\ Act\ one\ \{-\}} (followed by \texttt{\#\ A\ chapter})

Add an unnumbered part: \texttt{\#\ (PART\textbackslash{}*)\ Act\ one\ \{-\}} (followed by \texttt{\#\ A\ chapter})

Add an appendix as a special kind of un-numbered part: \texttt{\#\ (APPENDIX)\ Other\ stuff\ \{-\}} (followed by \texttt{\#\ A\ chapter}). Chapters in an appendix are prepended with letters instead of numbers.

\hypertarget{footnotes-and-citations}{%
\chapter{Footnotes and citations}\label{footnotes-and-citations}}

\hypertarget{footnotes}{%
\section{Footnotes}\label{footnotes}}

Footnotes are put inside the square brackets after a caret \texttt{\^{}{[}{]}}. Like this one \footnote{This is a footnote.}.

\hypertarget{citations}{%
\section{Citations}\label{citations}}

Reference items in your bibliography file(s) using \texttt{@key}.

For example, we are using the \textbf{bookdown} package \citep{R-bookdown} (check out the last code chunk in index.Rmd to see how this citation key was added) in this sample book, which was built on top of R Markdown and \textbf{knitr} \citep{xie2015} (this citation was added manually in an external file book.bib).
Note that the \texttt{.bib} files need to be listed in the index.Rmd with the YAML \texttt{bibliography} key.

The RStudio Visual Markdown Editor can also make it easier to insert citations: \url{https://rstudio.github.io/visual-markdown-editing/\#/citations}

\hypertarget{blocks}{%
\chapter{Blocks}\label{blocks}}

\hypertarget{equations}{%
\section{Equations}\label{equations}}

Here is an equation.

\begin{equation} 
  f\left(k\right) = \binom{n}{k} p^k\left(1-p\right)^{n-k}
  \label{eq:binom}
\end{equation}

You may refer to using \texttt{\textbackslash{}@ref(eq:binom)}, like see Equation \eqref{eq:binom}.

\hypertarget{theorems-and-proofs}{%
\section{Theorems and proofs}\label{theorems-and-proofs}}

Labeled theorems can be referenced in text using \texttt{\textbackslash{}@ref(thm:tri)}, for example, check out this smart theorem \ref{thm:tri}.

\begin{theorem}
\protect\hypertarget{thm:tri}{}\label{thm:tri}For a right triangle, if \(c\) denotes the \emph{length} of the hypotenuse
and \(a\) and \(b\) denote the lengths of the \textbf{other} two sides, we have
\[a^2 + b^2 = c^2\]
\end{theorem}

Read more here \url{https://bookdown.org/yihui/bookdown/markdown-extensions-by-bookdown.html}.

\hypertarget{callout-blocks}{%
\section{Callout blocks}\label{callout-blocks}}

The R Markdown Cookbook provides more help on how to use custom blocks to design your own callouts: \url{https://bookdown.org/yihui/rmarkdown-cookbook/custom-blocks.html}

\hypertarget{sharing-your-book}{%
\chapter{Sharing your book}\label{sharing-your-book}}

\hypertarget{publishing}{%
\section{Publishing}\label{publishing}}

HTML books can be published online, see: \url{https://bookdown.org/yihui/bookdown/publishing.html}

\hypertarget{pages}{%
\section{404 pages}\label{pages}}

By default, users will be directed to a 404 page if they try to access a webpage that cannot be found. If you'd like to customize your 404 page instead of using the default, you may add either a \texttt{\_404.Rmd} or \texttt{\_404.md} file to your project root and use code and/or Markdown syntax.

\hypertarget{metadata-for-sharing}{%
\section{Metadata for sharing}\label{metadata-for-sharing}}

Bookdown HTML books will provide HTML metadata for social sharing on platforms like Twitter, Facebook, and LinkedIn, using information you provide in the \texttt{index.Rmd} YAML. To setup, set the \texttt{url} for your book and the path to your \texttt{cover-image} file. Your book's \texttt{title} and \texttt{description} are also used.

This \texttt{gitbook} uses the same social sharing data across all chapters in your book- all links shared will look the same.

Specify your book's source repository on GitHub using the \texttt{edit} key under the configuration options in the \texttt{\_output.yml} file, which allows users to suggest an edit by linking to a chapter's source file.

Read more about the features of this output format here:

\url{https://pkgs.rstudio.com/bookdown/reference/gitbook.html}

Or use:

\begin{Shaded}
\begin{Highlighting}[]
\NormalTok{?bookdown}\SpecialCharTok{::}\NormalTok{gitbook}
\end{Highlighting}
\end{Shaded}


  \bibliography{STATSnMETH.bib}

\end{document}
